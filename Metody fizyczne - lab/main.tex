\documentclass[a4paper,12pt]{article}
\usepackage[MeX]{polski}
\usepackage{fontenc}
\usepackage[utf8]{inputenc}
\usepackage{amsmath,amsfonts}
\usepackage{graphicx}
\usepackage[usenames,dvipsnames]{xcolor}
\usepackage[small]{caption}
\usepackage{makecell} % Used to break text in a single cell of a table.
\usepackage{url}
\usepackage{multirow}
\def\UrlBreaks{\do\/\do-\/\do.\/\do_}
\usepackage[all]{nowidow}
\usepackage[shortlabels]{enumitem}
\usepackage{upquote}
\usepackage{setspace}
\setstretch{1.0} % Tu sterujemy liczbowo odstępem między liniami tekstu.
\usepackage[top=25mm,left=25mm,bottom=25mm,right=25mm]{geometry}
\usepackage[colorlinks=true,citecolor=blue,linkcolor=blue,urlcolor=blue]{hyperref}
\usepackage{siunitx}
\usepackage{hyperref}
\usepackage{float}
\usepackage{biblatex}
\usepackage{lipsum}
\usepackage{subcaption}
\usepackage{placeins}
% tikz libraries
\usepackage{tikz}
\usetikzlibrary{calc}
\usetikzlibrary{arrows.meta}





\begin{document}
	
	%\pagenumbering{arabic}
	\begin{center}
		{\huge \bf Wpływ muzyki na reakcje fizjologiczne}
		\vspace{.3in}
		
		{\Large{Student: Jan Nowak}} \\
		{\large{Indeks: 268357}} \\
            {\Large{Student: Szymon Kubica}} \\
		{\large{Indeks: 264068}} \\
		
	\end{center}
	
	\section{Cel eksperymentu}
Celem eksperymentu jest sprawdzenie, jak różne rodzaje muzyki wpływają na parametry fizjologiczne, takie jak tętno, aktywność mięśniowa, przewodnictwo skóry oraz ruch ciała - w zależności od dostępności sprzętu pomiarowego.

\section{Metodologia eksperymentu}
\subsection{Uczestnicy}
\begin{itemize}
    \item Grupa min. 6 osób (różny wiek, płeć)
    \item Brak zaburzeń neurologicznych lub kardiologicznych (dla wiarygodności wyników)
   
\end{itemize}

\subsection{Rodzaje muzyki}
Podział na 3--5 różnych gatunków:
\begin{itemize}
    \item \textbf{Klasyczna} (np. Mozart) -- relaksacyjna
    \item \textbf{Rock/metal} (np. AC/DC) -- pobudzająca
    \item \textbf{Elektroniczna} (np. techno) -- rytmiczna
    \item \textbf{Ambient} (np. lo-fi, chillout) -- neutralna
    \item \textbf{Brak muzyki} (cisza) -- próba kontrolna 
\end{itemize}
Każdy uczestnik słucha krótkich (2--3 min) fragmentów każdej kategorii muzycznej w losowej kolejności. Pomiędzy próbami 1--2 min ciszy na normalizację reakcji.

\subsection{Rejestrowane sygnały biomedyczne}
\begin{itemize}
    \item \textbf{EKG} -- rytm serca (HR) i zmienność rytmu serca (HRV)
    \item \textbf{EMG} -- napięcie mięśni twarzy - czoła i szczęki
    \item \textbf{Częstotliwość respiracji} -- ruchy klatki piersiowej
\end{itemize}

\subsection{Warunki eksperymentalne}
\begin{itemize}
    \item Ciche pomieszczenie
    \item Słuchawki dobrej jakości (brak wpływu dźwięków otoczenia)
    \item Stała głośność muzyki (np. 70 dB)
    \item Niskie oświetlenie pomieszczenia
    \item Pozycja siedząca podczas badań
\end{itemize}

\section{Analiza danych}
\subsection{Przetwarzanie sygnałów}
\begin{itemize}
    \item \textbf{EKG} -- obliczenie tętna i zmienności rytmu (HRV) -- za pomocą fotopletyzmografii
    \item \textbf{EMG} -- analiza napięcia mięśniowego
    \item \textbf{Akcelerometr} -- analiza ruchów w rytm muzyki
\end{itemize}

\subsection{Porównania statystyczne}
\begin{itemize}
    \item Testy ANOVA -- porównanie średnich wartości dla różnych rodzajów muzyki
    \item Korelacje -- analiza zależności między HRV a rodzajem muzyki
    \item Analiza klastrów -- grupowanie uczestników na podstawie reakcji
\end{itemize}

\subsection{Wizualizacja danych}
\begin{itemize}
    \item Wykresy czasowe (np. zmiany tętna w czasie słuchania muzyki)
    %\item Heatmapy aktywności mięśni
    \item Analiza częstości występowania reakcji dla różnych gatunków muzyki
\end{itemize}

\section{Możliwe wnioski}
\begin{itemize}
    \item Czy muzyka klasyczna faktycznie uspokaja?
    \item Czy szybkie tempo (np. rock, techno) powoduje wzrost tętna?
    \item Czy emocjonalna muzyka wpływa na przewodnictwo skóry i napięcie mięśniowe?
    \item Czy ludzie intuicyjnie poruszają się w rytm muzyki?
\end{itemize}

%\section{Rozszerzenia projektu}
%\begin{itemize}
%    \item Porównanie reakcji między grupami (np. muzycy vs. osoby bez wykształcenia muzycznego)
%    \item Zastosowanie EEG do pomiaru aktywności mózgowej
%    \item Analiza reakcji w rzeczywistych warunkach (np. koncert vs. laboratorium)
%\end{itemize}

\section{Przegląd literatury}
Badania wskazują, że muzyka ma istotny wpływ na reakcje fizjologiczne organizmu. W szczególności:
\begin{itemize}
    \item Szybkie tempo muzyki może zwiększać tętno i przewodnictwo skóry, wskazując na wyższy poziom pobudzenia \cite{PMC9417331}.
    \item Wolniejsza muzyka ma działanie relaksujące, wpływając na zmniejszenie napięcia mięśniowego \cite{PMC11411295}.
    \item Indywidualne preferencje muzyczne mogą modyfikować reakcje fizjologiczne \cite{PMC9417331}.
\end{itemize}


\begin{thebibliography}{9}
\bibitem{PMC9417331} \href{https://pmc.ncbi.nlm.nih.gov/articles/PMC9417331/}{Artykuł PMC9417331}
\bibitem{PMC11411295} \href{https://pmc.ncbi.nlm.nih.gov/articles/PMC11411295/}{Artykuł PMC11411295}
\end{thebibliography}


\end{document}